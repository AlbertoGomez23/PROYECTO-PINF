%\documentclass[10pt]{amsart}
\documentclass[11pt,a4paper,t]{beamer}

%%%%%%%%%%%%%%%%%%%%%%%%%%%%%%%%%%%%%%%
\usepackage{amsmath,amssymb}
%\usepackage{amsthm} % LO CARGA BEAMER AUTOM\'{A}TICAMENTE
\usepackage[spanish]{babel}
\usepackage[latin1]{inputenc}

%\usepackage[T1]{fontenc}
%\usepackage{lmodern}

%% Para cargar una pel\'{\i}cula:
\usepackage{multimedia}

%% Para hacer dibujos:
\usepackage{tikz}

\setbeamertemplate{background canvas}[vertical shading][bottom=red!10,top=blue!10]

\usetheme{Warsaw}
% OPCIONES: default, boxes, Bergen, Boadilla, Madrid, Pittsburgh
% OPCIONES: Rochester, Antibes, JuanLesPins, Montpellier, Berkeley,
% OPCIONES: PaloAlto, Goettingen, Marburg, Hannover, Berlin, Ilmenau,
% OPCIONES: Dresden, Darmstadt, Frankfurt, Singapore, Szeged,
% OPCIONES: Copenhagen, Luebeck, Malmoe, Warsaw
% Una contribuida por alguien: JLTree

\useinnertheme{default} % Adem\'{a}s, muchos tienen opciones
% OPCIONES: default, circles, rectangles, rounded, inmargin

\useoutertheme{default} % Adem\'{a}s, muchos tienen opciones
% OPCIONES: default, infolines, miniframes, smoothbars
% OPCIONES: sidebar, split, shadow, tree, smoothtree

\usecolortheme{whale} % Adem\'{a}s, muchos tienen opciones
% OPCIONES: default, structure, sidebartab
% OPCIONES (complete color) : albatross, beetle, crane, dove
% OPCIONES (B/N): dove, fly, seagull
% OPCIONES (more complete color): wolverine, beaver
% OPCIONES (inner color): lily, orchid, rose
% OPCIONES (outer color): whale, seahorse, dolphin

\usefonttheme{default} % Adem\'{a}s, muchos tienen opciones
% OPCIONES: default, professionalfonts, serif
% OPCIONES: structurebold, structureitalicserif, structuresmallcapsserif

\setbeamercovered{transparent}
% OPCIONES: invisible, transparent, dynamic, highly dinamic

% Para que no salgan los "navegadores del pdf" en cada p\'{a}gina de un beamer:
\setbeamertemplate{navigation symbols}{}

%%%%%%%%%%%%%%%%%%%%%%%%%%% T\'{I}TULO
\title{C\'{a}lculo y representaci\'{o}n de eclipses de Sol}

\author[F. Baeza Richer]{Federico Baeza Richer}

\institute[M\'{a}ster en Modelizaci\'{o}n Matem\'{a}tica, Estad\'{\i}stica y Computaci\'{o}n]{M\'{a}ster en Modelizaci\'{o}n Matem\'{a}tica, Estad\'{\i}stica y Computaci\'{o}n}

\date{2 de julio de 2010}

\begin{document}
%%%%%%%%%%%%%%%%%%%%%%%%%%% 1 : T\'{I}TULO
\begin{frame}
\titlepage
\end{frame}

%%%%%%%%%%%%%%%%%%%%%%%%%%% 2: \'{I}NDICE
\section{Motivaci\'{o}n del trabajo}


\begin{frame}
  \frametitle{\'{I}ndice}
  \tableofcontents%[pausesections]
\end{frame}




%%%%%%%%%%%%%%%%%%%%%%%%%%%



\begin{frame}

\frametitle{Motivaci\'{o}n del trabajo}
\begin{itemize}
\item A propuesta del Jefe de la secci\'{o}n de Efem\'{e}rides del Real Observatorio de la Armada.
\item Aprovechar la capacidad de un procesador simb\'{o}lico como \textsl{Mathematica}, para el c\'{a}lculo de los par\'{a}metros y curvas de un eclipse de Sol.
\item Formulaci\'{o}n elegante y sencilla frente a complejidad de los m\'{e}todos cl\'{a}sicos (anal\'{\i}ticos y num\'{e}ricos).
\item Automatizar la representaci\'{o}n en mapas. C\'{a}lculo de puntos particulares.

\end{itemize}

\end{frame}

%%%%%%%%%%%%%%%%%%%%%%%%%%%


\section{Descripci\'{o}n}

%%%%%%%%%%%%%%%%%%%%%%%%%%%%%

\subsection{Tipos de eclipses}

\begin{frame}

\frametitle{Tipos de eclipses de Sol}

\only<1>{

  \begin{itemize}
    \item Condiciones eclipses de Sol
    \item Cono de sombra, de penumbra y eje de la sombra
  \end{itemize}
  \begin{center}
     \includegraphics[width=0.8\textwidth]{./Figuras/Conos.pdf}
  \end{center}
  \begin{itemize}
    \item Modelo IAU (Uni\'{o}n Astron\'{o}mica Internacional)
  \end{itemize}}

\only<2>{Divisi\'{o}n cl\'{a}sica de los eclipses: parcial (arriba), total (centro) y anular (abajo)

    \begin{center}
      \includegraphics[width=0.54\textwidth]{./Figuras/Tipos.pdf}
    \end{center}}

\only<3>{En este trabajo: parcial (arriba, izda), no central (arriba, dcha), central con una sola curva de totalidad (abajo, izda), central con una sola curva de parcialidad (abajo, ctro) y central con las dos curvas de parcialidad (abajo, dcha).
  \begin{center}
  \includegraphics[width=0.74\textwidth]{./Figuras/Sombras.pdf}
  \end{center}}


\end{frame}


%%%%%%%%%%%%%%%%%%%%%%%%%%% 8

\subsection{Par\'{a}metros y curvas caracter\'{\i}sticos}

\begin{frame}
\frametitle{Circunstancias generales del eclipse}
Destacan las siguientes:
\begin{itemize}
  \item Principio y fin del eclipse.
  \item Principio y fin del eclipse total/anular.
  \item Principio y fin de la curva central del eclipse.
  \item M\'{a}ximo del eclipse.
  \item Magnitud del eclipse.
\end{itemize}
\end{frame}

\begin{frame}

\frametitle{Mapa del Eclipse}
\only<1-8>{\center{Eclipse central con las dos curvas de parcialidad}}
\only<9>{\center{Eclipse central con una sola curva de parcialidad}}
\only<10>{\center{Eclipse central con una sola curva de totalidad}}
\only<11>{\center{Eclipse parcial}}
\begin{overprint}
\onslide<1>
  \vspace{0.8cm}
  \begin{center}
  \includegraphics[scale=0.8]{./Figuras/s1.pdf}
  \end{center}
\onslide<2>
  \begin{itemize}
    \item Primer y \'{u}ltimo contactos
  \end{itemize}
  \begin{center}
  \includegraphics[scale=0.8]{./Figuras/s2.pdf}
  \end{center}
\onslide<3>
  \begin{itemize}
    \item Curvas de contacto en el horizonte
  \end{itemize}
  \begin{center}
  \includegraphics[scale=0.8]{./Figuras/s3.pdf}
  \end{center}
\onslide<4>
  \begin{itemize}
    \item Curvas de m\'{a}ximo en el horizonte
  \end{itemize}
  \begin{center}
  \includegraphics[scale=0.8]{./Figuras/s4.pdf}
  \end{center}
\onslide<5>
  \begin{itemize}
    \item Curvas l\'{\i}mite Norte y Sur de la penumbra
  \end{itemize}
  \begin{center}
  \includegraphics[scale=0.8]{./Figuras/s5.pdf}
  \end{center}
\onslide<6>
  \begin{itemize}
    \item Curvas l\'{\i}mite Norte y Sur de la sombra
  \end{itemize}
  \begin{center}
  \includegraphics[scale=0.8]{./Figuras/s6.pdf}
  \end{center}
\onslide<7>
  \begin{itemize}
    \item Curva de la centralidad
  \end{itemize}
  \begin{center}
  \includegraphics[scale=0.8]{./Figuras/s7.pdf}
  \end{center}
\onslide<8>
  \begin{itemize}
    \item Curvas de simultaneidad de principio y fin del eclipse
  \end{itemize}
  \begin{center}
  \includegraphics[scale=0.8]{./Figuras/s8.pdf}
  \end{center}
\onslide<9>
  \vspace{0.8cm}
  \begin{center}
  \includegraphics[scale=0.48]{./Figuras/01ago2008.pdf}
  \end{center}
\onslide<10>
  \vspace{0.8cm}
  \begin{center}
  \includegraphics[scale=0.7]{./Figuras/31may2003.pdf}
  \end{center}
\onslide<11>
  \vspace{0.8cm}
  \begin{center}
  \includegraphics[scale=0.65]{./Figuras/19abr2004.pdf}
  \end{center}

\transblindshorizontal<9,10,11>
\end{overprint}
\end{frame}

%%%%%%%%%%%%%%%%%%%%%%%%%%% 9

\subsection{Sistema fundamental de coordenadas}

\begin{frame}

\frametitle{Sistema fundamental de coordenadas}

\only<1,2>{
\begin{center}
\includegraphics[width=0.6\textwidth]{./Figuras/Sistemas.pdf}
\end{center}}

\begin{overprint}

\onslide<1>
\begin{itemize}
\item Definici\'{o}n
\item Plano y elipse fundamental
\item Plano del observador
\end{itemize}

\onslide<2>
Elementos besselianos
\begin{itemize}
\item Definici\'{o}n
\item $x, y, d, \mu, l_{p}, l_{s}, i_{p}, i_{s}$
\end{itemize}



\onslide<3>
Las coordenadas en el sistema fundamental $(\xi,\eta,\zeta)$, de un punto de coordenadas geogr\'{a}ficas $(\lambda,\phi)$, ser\'an:
\[
\begin{array}{r@{{}={}}l}
\xi   & \rho \cos\phi \sen h, \\
\eta  & \rho \sen\phi \cos d - \rho \cos\phi\sen d \cos h, \\
\zeta & \rho \sen\phi \sen d + \rho \cos\phi\cos d \cos h,
\end{array}
\]
donde $h=\mu + \lambda$  es el horario local del eje de la sombra.

\vspace{0.5cm}

Sus derivadas con respecto al tiempo son:
\[
\begin{array}{r@{{}={}}l}
\xi' & \mu'(\zeta\cos d-\eta\sen d),\\
\eta' & -\zeta d'+\mu'\xi\sen d,\\
\zeta' & \eta d'-\mu'\xi\cos d.
\end{array}
\]

\onslide<4>
Para un observador de coordenadas $(\xi,\eta,\zeta)$, el radio del cono de penumbra en el plano del observador ($L_p$), del de sombra ($L_s$), as\'{\i} como la distancia ($\Delta$) al eje de la sombra, est\'an definidos por:
\begin{equation}\label{LpLsD}
\begin{array}{r@{{}={}}l}
L_{p} & l_{p}-\zeta i_{p},\\[1ex]
L_{s} & l_{s}-\zeta i_{s},\\[1ex]
\Delta & \sqrt{(x-\xi)^2+(y-\eta)^2}.
\end{array}
\end{equation}
\end{overprint}

\end{frame}

%%%%%%%%%%%%%%%%%%%%%%%%%%%

\section{Desarrollo del programa}

%%%%%%%%%%%%%%%%%%%%%%%%%%%%%%%%

\subsection{Implementaci\'{o}n algoritmos}
%\begin{frame}
%\frametitle{C\'{a}lculo elementos besselianos}
%Sean $\vec{r}_{\astrosun}$ y $\vec{r}_{\leftmoon}$ los vectores de posici\'{o}n geoc\'{e}ntricos del Sol y la Luna,
%la direcci\'on del eje $z$ est\'a definida por el vector de posici\'on del Sol respecto a la Luna $\vec{r}_{\leftmoon\kern-.5ex\hbox{-\astrosun}} = \vec{r}_{\astrosun}-\vec{r}_{\leftmoon}$, quedando determinados los elementos besselianos $d$ y $\mu$.\newline
%Las coordenadas del centro de la Luna en el sistema fundamental definen los elementos $x$ e $y$:
%\[
%\left (
%\begin{array}{c}
%  x = x_{\leftmoon} \\
%  y = y_{\leftmoon} \\
%  z_{\leftmoon}
%\end{array}
%\right )=
%{\cal R}^{\rm{T}}
%\left (
%\begin{array}{c}
%  X_{\leftmoon} \\
%  Y_{\leftmoon} \\
%  Z_{\leftmoon}
%\end{array}
%\right ),
%\]
%Los elementos $l_{p}, l_{s}, i_{p}, i_{s}$ se obtienen por trigonometr\'{\i}a.
%\end{frame}
\begin{frame}
\frametitle{Datos iniciales y elementos besselianos}
\begin{itemize}
 \item De las efem\'{e}rides b\'{a}sicas del Jet Propulsion Laboratory se obtiene una base de datos a intervalos de tiempo constante de lo siguiente:
 \begin{itemize}
  \item Ascensi\'{o}n recta, declinaci\'{o}n y distancia geoc\'{e}ntrica de Sol y Luna.
  \item Tiempo Sid\'{e}reo.
 \end{itemize}
 \item Por geometr\'{\i}a y trigonometr\'{\i}a se obtienen los elementos besselianos para los instantes de los datos.
 \item Adem\'{a}s de los elementos besselianos se requieren dos par\'{a}metros m\'{a}s:
 \begin{itemize}
  \item Valor del semieje menor de la elipse fundamental ($b$).
  \item Distancia del eje de sombra a la elipse fundamental ($q$).
 \end{itemize}
 \item De todos los elementos besselianos, as\'{\i} como de estos nuevos par\'{a}metros, se obtienen funciones interpolantes y sus derivadas mediante \texttt{Interpolation} y \texttt{Derivative}.
\end{itemize}
\end{frame}


%\begin{frame}
%\frametitle{Otras par\'{a}metros necesarios}
%Adem\'{a}s de los elementos besselianos se requieren dos funciones m\'{a}s:
%\begin{itemize}
%\item Valor del semieje de la elipse fundamental ($b$): Queda un\'{\i}vocamente determinado por el elemento $d$.
%\item Distancia del eje de sombra a la elipse fundamental ($q$): Se obtiene una vez definida la elipse fundamental y sabiendo la posici\'{o}n de eje de sombra (elementos $x$ e $y$).
%\end{itemize}
%De todos los elementos besselianos, as\'{\i} como de estos nuevos par\'{a}metros se obtienen funciones interpolantes y sus derivadas mediante \texttt{Interpolation} y \texttt{Derivative}.
%\end{frame}

\begin{frame}
\frametitle{Determinaci\'{o}n tipo de eclipse}
Se determinan en funci\'{o}n de lo siguiente:
\begin{itemize}
  \item Si m\'{\i}n($q-l_{p}>0$): no hay eclipse.
  \item Si m\'{\i}n($q-l_{p}\leq0$) y m\'{\i}n($q-l_{s}>0$): Eclipse parcial.
  \item Si m\'{\i}n($q-l_{s}\leq0$) y m\'{\i}n($q>0$): Eclipse no central.
  \item Si m\'{\i}n($q\leq0$) y m\'{\i}n($q+l_{s}>0$): Eclipse central con una sola curva de totalidad.
  \item Si m\'{\i}n($q+l_{s}\leq0$) y m\'{\i}n($q+l_{p}>0$): Eclipse central con una sola curva de parcialidad.
  \item Si m\'{\i}n($q+l_{p}\leq0$): Eclipse central con las dos curvas de parcialidad.
\end{itemize}
Se crean variables auxiliares l\'{o}gicas.
\end{frame}

\begin{frame}
\frametitle{Principio y fin de las distintas fases}
Utilizando la funci\'on \verb"FindRoot" el principio y fin de cada fase se obtiene calculando las horas para las que se anulan las siguientes funciones:
\begin{center}
\begin{tabular}{ll}
\hphantom{Contactos interiores del cono de penumbra olao}\\[-2ex]
Principio y fin del eclipse parcial:\dotfill & $q - l_p$\\[1ex]
Principio y fin del eclipse total/anular:\dotfill & $q - l_s$\\[1ex]
Principio y fin del eclipse central:\dotfill & $q$\\
\end{tabular}
\end{center}

\end{frame}

\begin{frame}
\frametitle{M\'{a}ximo del eclipse}
Es el instante de m\'{\i}nima distancia del eje de la sombra al centro de la Tierra. Se obtiene aplicando la funci\'{o}n  \verb"FindRoot" a la derivada de la expresi\'{o}n $\sqrt{x^2+y^2}$.

\end{frame}



\begin{frame}
\frametitle{Curva central del eclipse}
Se calcula para distintos instantes, el punto de corte del eje de sombra con el elipsoide. Se obtienen los puntos resolviendo, mediante la funci\'{o}n \verb"NSolve", el sistema de ecuaciones:
\[
\begin{array}{r@{{}={}}l}
\xi & x,\\
\eta & y,\\
X^2+Y^2+\dfrac{Z^2}{(1-f)^2} & 1,
\end{array}
\]
donde $x$ e $y$ son los valores correspondientes al instante de m\'aximo, y $(X, Y,Z)$ deben expresarse en coordenadas fundamentales ($\xi,\eta,\zeta$).

\end{frame}

\begin{frame}
\frametitle{Curvas de contacto en el horizonte, de m\'{a}ximo en el horizonte y de simultaneidad de principio y fin del eclipse}
Se obtienen, igual que para la curva anterior, resolviendo respectivamente los siguientes sistemas:
\begin{columns}[t]
 \begin{column}{0.27\textwidth}
  \[
  \left .
  \begin{array}{r@{{}={}}l}
  \zeta & 0, \\[1.5ex]
  \dfrac{\eta^2}{b^2}+\xi^2 & 1,\\[2.2ex]
  \Delta & l_{p},
  \end{array}
  \right \}
  \]
 \end{column}
 \begin{column}{0.27\textwidth}
  \[
  \left .
  \begin{array}{r@{{}={}}l}
  \zeta & 0, \\[1.5ex]
  \dfrac{\eta^2}{b^2}+\xi^2 & 1,\\[1.8ex]
  \dfrac{d(\Delta)}{dt} & 0,
  \end{array}
  \right \}
  \]
 \end{column}
 \begin{column}{0.46\textwidth}
  \[
  \left .
  \begin{array}{r@{{}={}}l}
  \zeta & {\rm cte.}, \\[2ex]
  \Delta & L_{p},\\[1.8ex]
  X^2+Y^2+\dfrac{Z^2}{(1-f)^2} & 1.
  \end{array}
  \right \}
  \]
 \end{column}
\end{columns}
\end{frame}

%\begin{frame}
%\frametitle{Curvas de contacto en el horizonte}
%Las conforman, para cada instante, los puntos de la elipse fundamental, para los cuales la distancia al eje de sombra coincide con el radio del cono de penumbra en el plano fundamental. Para obtenerlos, se resuelve mediante la funci\'{o}n \verb"NSolve" el siguiente sistema de ecuaciones, del que se desechan las soluciones no reales:
%\[
%\begin{array}{r@{{}={}}l}
%\zeta & 0, \\
%\dfrac{\eta^2}{b^2}+\xi^2 & 1,\\
%\Delta & l_{p},
%\end{array}
%\]
%donde $b$ es el semieje menor de la elipse fundamental.
%\end{frame}

%\begin{frame}
%\frametitle{Curvas de m\'{a}ximo en el horizonte}
%Las conforman, para cada instante, los puntos de la elipse fundamental que se encuentran a su m\'{\i}nima distancia del eje de la sombra. Para ellos pues, la funci\'{o}n $\Delta$ tiene un m\'{\i}nimo. Para obtenerlos, se resuelve mediante la funci\'{o}n \verb"NSolve" el siguiente sistema de ecuaciones, del que se mantienen exclusivamente los puntos reales y que cumplan adem\'{a}s $l_{p}-\Delta\geq0$, ya que de no ser as\'{\i}, se econtrar\'{\i}an fuera del cono de penumbra y por lo tanto no ver\'{\i}an el eclipse en ese instante.
%\[
%\begin{array}{r@{{}={}}l}
%\zeta & 0, \\
%\dfrac{\eta^2}{b^2}+\xi^2 & 1,\\
%\dfrac{d(\Delta)}{dt} & 0.
%\end{array}
%\]
%\end{frame}

\begin{frame}
\frametitle{Curvas l\'{\i}mite Norte y Sur}
\only<1>{Seg\'{u}n la bibliograf\'{\i}a son las curvas formadas por los puntos de coordenada $\zeta\geq0$, para los que el eclipse se reduce a una \'{u}nica tangencia de los discos del Sol y de la Luna. Para ellos la funci\'{o}n $L-\Delta$ (donde $L$ ser\'{a} $L_{s}$ o $L_{p}$ seg\'{u}n se trate del cono de sombra o de penumbra) se anula y tiene un m\'{a}ximo.
\medskip

Se resuelve el siguiente sistema de ecuaciones:
\[
\begin{array}{r@{{}={}}l}
\dfrac{d(L-\Delta)}{dt} & 0, \\[1.5ex]
L-\Delta & 0, \\[1.5ex]
X^2+Y^2+\dfrac{Z^2}{(1-f)^2} & 1.
\end{array}
\]}
\only<2>{

\begin{center}
\includegraphics[width=0.45\textwidth]{./Figuras/criticocompleto.pdf}
\quad
\includegraphics[width=0.45\textwidth]{./Figuras/criticozoomsin.pdf}\\
\end{center}
Vista ampliada (derecha) de la zona de uni\'{o}n de las curvas de contacto en el horizonte para un eclipse parcial (izquierda).
}
\only<3>{
\begin{columns}[c]
 \begin{column}{0.5\textwidth}
  \begin{center}
   \includegraphics[width=0.95\textwidth]{./Figuras/criticozoomcon.pdf}
  \end{center}
 \end{column}
 \begin{column}{0.5\textwidth}
  La nueva curva se calcula hallando los puntos que maximizan y anulan a la vez la funci\'{o}n $\zeta$. Son la soluci\'{o}n del siguiente sistema:
  \[
  \begin{array}{r@{{}={}}l}
  \zeta & 0, \\[1.2ex]
  \dfrac{\eta^2}{b^2}+\xi^2 & 1,\\[1.6ex]
  \dfrac{d(\zeta)}{dt} & 0.
  \end{array}
  \]
 \end{column}
\end{columns}
}

\end{frame}

\begin{frame}
\frametitle{Puntos de coincidencia de las curvas de m\'{a}ximo en el horizonte con las l\'{\i}mite Norte y Sur}

Ejemplo de mapas sin haber calculado estos puntos.
\begin{center}
  \includegraphics[width=0.43\textwidth]{./Figuras/11jul2010c5.pdf}
  \quad
  \includegraphics[width=0.43\textwidth]{./Figuras/11jul2010c1.pdf}
  \newline
  \bigskip
  {\small Puntos cada cinco minutos \qquad \qquad Puntos cada minuto \qquad \qquad}
\end{center}
Se calcula, mediante \verb"FindRoot", el instante en que los puntos de la curva de m\'{a}ximo en el horizonte tienen nula su distancia al borde del cono.

\end{frame}

%\begin{frame}
%\frametitle{Curvas de simultaneidad de principio y fin de eclipse}
%Son los puntos, para cada instante, de corte del cono de penumbra con el elipsoide terrestre. El m\'{e}todo que se aplica es ir calculando puntos de estas curvas en distintos planos $\zeta={\rm cte.}$, para valores $0\leq\zeta\leq1$. Los puntos que se obtienen son los de corte de una circunferencia y una elipse, que son las figuras geom\'{e}tricas que forman el cono y el elipsoide, respectivamente, en cada plano. Para obtenerlos se resuelve, utilizando la funci\'{o}n \verb"NSolve", el siguiente sistema de ecuaciones, del que se desechan las soluciones no reales.
%\[
%\begin{array}{r@{{}={}}l}
%\zeta & {\rm cte.}, \\[1ex]
%\Delta & L_{p},\\[1.5ex]
%X^2+Y^2+\dfrac{Z^2}{(1-f)^2} & 1.
%\end{array}
%\]
%\end{frame}

%%%%%%%%%%%%%%%%%%%%%%%%%

\subsection{Dibujo del mapa}

\begin{frame}
\frametitle{Ordenado de puntos y tipo de representaci\'{o}n}
\begin{itemize}
\item Ordenado de puntos de una misma curva y distribuci\'{o}n en sus curvas respectivas.
\item Obtenci\'{o}n de curvas mediante la funci\'{o}n \verb"BSpline".
\item Representaci\'{o}n estereogr\'{a}fica tangente de las curvas, junto con ret\'{\i}cula y continentes.
\end{itemize}
\end{frame}


\begin{frame}
\frametitle{Mapas}
\begin{overprint}

\only<1>{\center{Eclipse anular de 26 de enero de 2009}
\bigskip}

\only<2>{\center{Eclipse parcial de 19 de abril de 2004}
\bigskip}


\onslide<1>
\begin{center}
\includegraphics[width=0.42\textwidth]{./Figuras/EjemploMapaAnular.pdf}
\qquad
\includegraphics[width=0.4\textwidth]{./Figuras/s1.pdf}\\
\medskip
{\small Mapa almanaque \qquad \qquad \qquad \qquad Mapa programa \qquad}
\end{center}


\onslide<2>
\begin{center}
\includegraphics[width=0.44\textwidth]{./Figuras/EjemploMapaParcial2.pdf}
\quad
\includegraphics[width=0.4\textwidth]{./Figuras/19abr2004.pdf}\\
\medskip
{\small Mapa almanaque \qquad \qquad \qquad \qquad Mapa programa \qquad}
\end{center}

\end{overprint}

\end{frame}

%%%%%%%%%%%%%%%%%%%%%%%%%%%

\subsection{Conclusiones}

\begin{frame}

\frametitle{Conclusiones}

\begin{itemize}
    \item Utilidad de procesador simb\'{o}lico para obtenci\'{o}n de curvas y par\'{a}metros mediante ecuaciones sencillas y claras.
    \item Muy buena herramienta para el dibujo.
    \item Se completa el dibujo con la obtenci\'{o}n de puntos caracter\'{\i}sticos.
    \item Hallazgo de nueva curva l\'{\i}mite en zonas cr\'{\i}ticas de algunos eclipses.
    \item Utilidad del programa para la secci\'{o}n de Efem\'{e}rides del ROA.
\end{itemize}

\end{frame}


\end{document}

%\documentclass[10pt]{amsart}
\documentclass[11pt,a4paper,t]{beamer}

%%%%%%%%%%%%%%%%%%%%%%%%%%%%%%%%%%%%%%%
\usepackage{amsmath,amssymb}
%\usepackage{amsthm} % LO CARGA BEAMER AUTOM\'{A}TICAMENTE
\usepackage[spanish]{babel}
\usepackage[latin1]{inputenc}

%\usepackage[T1]{fontenc}
%\usepackage{lmodern}

%% Para cargar una pel\'{\i}cula:
\usepackage{multimedia}

%% Para hacer dibujos:
\usepackage{tikz}

\setbeamertemplate{background canvas}[vertical shading][bottom=red!10,top=blue!10]

\usetheme{Warsaw}
% OPCIONES: default, boxes, Bergen, Boadilla, Madrid, Pittsburgh
% OPCIONES: Rochester, Antibes, JuanLesPins, Montpellier, Berkeley,
% OPCIONES: PaloAlto, Goettingen, Marburg, Hannover, Berlin, Ilmenau,
% OPCIONES: Dresden, Darmstadt, Frankfurt, Singapore, Szeged,
% OPCIONES: Copenhagen, Luebeck, Malmoe, Warsaw
% Una contribuida por alguien: JLTree

\useinnertheme{default} % Adem\'{a}s, muchos tienen opciones
% OPCIONES: default, circles, rectangles, rounded, inmargin

\useoutertheme{default} % Adem\'{a}s, muchos tienen opciones
% OPCIONES: default, infolines, miniframes, smoothbars
% OPCIONES: sidebar, split, shadow, tree, smoothtree

\usecolortheme{whale} % Adem\'{a}s, muchos tienen opciones
% OPCIONES: default, structure, sidebartab
% OPCIONES (complete color) : albatross, beetle, crane, dove
% OPCIONES (B/N): dove, fly, seagull
% OPCIONES (more complete color): wolverine, beaver
% OPCIONES (inner color): lily, orchid, rose
% OPCIONES (outer color): whale, seahorse, dolphin

\usefonttheme{default} % Adem\'{a}s, muchos tienen opciones
% OPCIONES: default, professionalfonts, serif
% OPCIONES: structurebold, structureitalicserif, structuresmallcapsserif

\setbeamercovered{transparent}
% OPCIONES: invisible, transparent, dynamic, highly dinamic

% Para que no salgan los "navegadores del pdf" en cada p\'{a}gina de un beamer:
\setbeamertemplate{navigation symbols}{}

%%%%%%%%%%%%%%%%%%%%%%%%%%% T\'{I}TULO
\title{C\'{a}lculo y representaci\'{o}n de eclipses de Sol}

\author[F. Baeza Richer]{Federico Baeza Richer}

\institute[M\'{a}ster]{M\'{a}ster}

\date{2 de julio de 2002}

\begin{document}
%%%%%%%%%%%%%%%%%%%%%%%%%%% 1 : T\'{I}TULO
\begin{frame}
\titlepage
\end{frame}

%%%%%%%%%%%%%%%%%%%%%%%%%%% 2: \'{I}NDICE
\section{\'Indice}
\begin{frame}
  \frametitle{\'Indice}
  \tableofcontents%[pausesections]
\end{frame}

\section{Ejemplos}
%%%%%%%%%%%%%%%%%%%%%%%%%%% 3
\subsection{Primeros ejemplos}

\begin{frame}

\frametitle{La primera pantalla}

Esto es un ejemplo de una trasparencia sencilla. Se escribe como en \LaTeX, tanto el texto como las f\'{o}rmulas:
\[
   \sum_{n=1}^\infty \frac{1}{n^2} = \frac{\pi^2}{6}
\]
Esta primera transparencia no tiene ning\'{u}n efecto especial:
\begin{itemize}
\item No hay colores
\item Todo el texto sale de una vez
\item El formato (cabecera, pie de p\'{a}gina, colores\dots) es autom\'{a}tico
\end{itemize}

\end{frame}

%%%%%%%%%%%%%%%%%%%%%%%%%%% 4

\begin{frame}

\frametitle{La segunda pantalla}

Resaltar un texto con color \alert{rojo} es muy f\'{a}cil.

Tambi\'{e}n se puede resaltar una f\'{o}rmula, o parte de ella:
\[
   \int_0^{+\infty} \frac{dx}{1 + x^2} = \frac{\alert{\pi}}{2}
\]

Y hacer que la transparencia salga en varias veces (a golpe de rat\'{o}n) tambi\'{e}n es sencillo:
\begin{itemize}
\item<2-> Esto sale en la segunda parte
\item<3-> Y esto en la tercera
\item<4-> Esto en la cuarta
\item<4-> Y esto tambi\'{e}n en la cuarta
\end{itemize}

\only<4->{Y no hace falta que sean apartados de una lista, aunque entonces la orden es otra.}

\end{frame}

%%%%%%%%%%%%%%%%%%%%%%%%%%% 5

\subsection{Otros efectos sencillos}

\begin{frame}

\frametitle{La tercera pantalla}

Se puede hacer que parte del texto aparezca y luego desaparezca:

\only<2>{\alert{Atenci\'{o}n a esta l\'{\i}nea, que va a desaparecer.}}

\[
   \int_0^{+\infty} \frac{dx}{1 + x^2} = \frac{\pi}{2}
\]

\only<3->{La l\'{\i}nea de arriba ha desaparecido.\bigskip}

\only<4>{
Hay una orden para recuadrar un texto o f\'{o}rmula:
\begin{block}{Teorema de Jacobo:}
Toda funci\'{o}n $f : \mathbb{R} \to \mathbb{R}$ continua es derivable.
\end{block}
}

\only<5-6>{\alert{Perd\'{o}n:} algo debe estar mal, porque he encontrado un contraejemplo, $f(x) = |x|$.}

\only<6>{\bigskip
!`Ah!\dots\ Ya me acuerdo. Es lo contrario:\dots}

\only<7>{
% Hay una orden para recuadrar un texto o f\'{o}rmula:
\begin{alertblock}{Teorema:}
Toda funci\'{o}n $f : \mathbb{R} \to \mathbb{R}$ derivable es continua.
\end{alertblock}
}

\end{frame}

%%%%%%%%%%%%%%%%%%%%%%%%%%% 6
\begin{frame}

\frametitle{La cuarta pantalla}

\begin{center}
\fbox{Una frase recuadrada}
\end{center}

Atenci\'{o}n, porque \alert<4-5>{esto} va a cambiar de color.

\only<2>{Pero todav\'{\i}a no.}
\only<3>{A\'{u}n no.}
\only<4->{!`Ahora!}
\only<5->{Ya hab\'{\i}a avisado.}
\only<6->{Ahora ha vuelto a su color original.}

\end{frame}
%%%%%%%%%%%%%%%%%%%%%%%%%%%%%

\begin{frame}

\frametitle{Tipos de eclipses}


\only<1>{Definiciones

  \begin{itemize}
    \item Cono de sombra, de penumbra y eje de la sombra
  \end{itemize}
  \begin{center}
    \includegraphics[width=0.8\textwidth]{./Figuras/Conos.pdf}
  \end{center}}

\only<2-4>{Divisi\'{o}n cl\'{a}sica de los eclipses

\begin{overprint}

\onslide<2>
    \begin{itemize}
      \item Eclipse parcial (arriba)
    \end{itemize}

\onslide<3>
    \begin{itemize}
      \item Eclipse total (centro)
    \end{itemize}

\onslide<4>
    \begin{itemize}
      \item Eclipse anular (abajo)
    \end{itemize}

\end{overprint}

\only<2-4>
  \begin{center}
  \includegraphics[width=0.5\textwidth]{./Figuras/Tipos.pdf}
  \end{center}}

\only<5-9>{Divisi\'{o}n de los eclipses en este trabajo

\begin{overprint}

\onslide<5>
    \begin{itemize}
      \item Eclipse parcial (arriba, izda)
    \end{itemize}

\onslide<6>
    \begin{itemize}
      \item Eclipse no central (arriba, dcha)
    \end{itemize}

\onslide<7>
    \begin{itemize}
      \item Eclipse central con una sola curva de totalidad (abajo, izda)
    \end{itemize}

\onslide<8>
    \begin{itemize}
      \item Eclipse central con una sola curva de parcialidad (abajo, ctro)
    \end{itemize}

\onslide<9>
    \begin{itemize}
      \item Eclipse central con las dos curvas de parcialidad (abajo, dcha)
    \end{itemize}

\end{overprint}

\only<5-9>
  \begin{center}
  \includegraphics[width=0.8\textwidth]{./Figuras/Sombras.pdf}
  \end{center}}



\end{frame}

%%%%%%%%%%%%%%%%%%%%%%%%%%% 7

\begin{frame}

\frametitle{La quinta pantalla}

A veces un texto peque\~{n}o se sustituye por uno grande, o al rev\'{e}s:

\only<1>{Un texto peque\~{n}o, que se va a sustituir por uno grande.}
\only<2->{Un texto grande:
\[
   \sum_{n=1}^\infty \frac{1}{n^2} = \frac{\pi^2}{6}.
\]
}

\alert<1-2>{Esto produce desplazamientos desagradables en la presentaci\'{o}n.}
\only<2->{Pero se puede arreglar.}

\begin{overprint}
\onslide<3>Un texto peque\~{n}o.
\onslide<4->Y uno grande:
\[
   \sum_{n=1}^\infty \frac{1}{n^2} = \frac{\pi^2}{6}.
\]
\end{overprint}

\only<3->{\alert{Hay una orden para que se guarde el hueco suficiente.}}

\end{frame}

%%%%%%%%%%%%%%%%%%%%%%%%%%% 8

\subsection{Par\'{a}metros y curvas caracter\'{\i}sticos}

\begin{frame}
\frametitle{Conjunci\'{o}n en ascensi\'{o}n recta}
Se dan los siguientes datos:
\begin{itemize}
  \item Hora de la conjunci\'{o}n.
  \pause
  \item Declinaci\'{o}n de Sol y Luna.
  \pause
  \item Paralaje horizontal ecuatorial de Sol y Luna.
  \pause
  \item Semidi\'{a}metro de Sol y Luna.
\end{itemize}
\end{frame}

\begin{frame}
\frametitle{Circunstancias generales del eclipse}
Destacan las siguientes:
\begin{itemize}
  \item Principio y fin del eclipse.
  \pause
  \item Principio y fin del eclipse total/anular.
  \pause
  \item Principio y fin de la curva central del eclipse.
  \pause
  \item M\'{a}ximo del eclipse.
  \pause
  \item Eclipse central al mediod\'{\i}a local.
  \pause
  \item Magnitud del eclipse.
\end{itemize}
\end{frame}

\begin{frame}

\frametitle{Mapa del Eclipse}
\only<1-8>{Eclipse central con las dos curvas de parcialidad}
\only<9>{Eclipse central con una sola curva de parcialidad}
\only<10>{Eclipse central con una sola curva de totalidad}
\only<11>{Eclipse parcial}
\begin{overprint}
\onslide<1>
  \vspace{0.8cm}
  \begin{center}
  \includegraphics[scale=0.8]{./Figuras/s1.pdf}
  \end{center}
\onslide<2>
  \begin{itemize}
    \item Primer y \'{u}ltimo contactos
  \end{itemize}
  \begin{center}
  \includegraphics[scale=0.8]{./Figuras/s2.pdf}
  \end{center}
\onslide<3>
  \begin{itemize}
    \item Curvas de contacto en el horizonte
  \end{itemize}
  \begin{center}
  \includegraphics[scale=0.8]{./Figuras/s3.pdf}
  \end{center}
\onslide<4>
  \begin{itemize}
    \item Curvas de m\'{a}ximo en el horizonte
  \end{itemize}
  \begin{center}
  \includegraphics[scale=0.8]{./Figuras/s4.pdf}
  \end{center}
\onslide<5>
  \begin{itemize}
    \item Curvas l\'{\i}mite Norte y Sur de la penumbra
  \end{itemize}
  \begin{center}
  \includegraphics[scale=0.8]{./Figuras/s5.pdf}
  \end{center}
\onslide<6>
  \begin{itemize}
    \item Curvas l\'{\i}mite Norte y Sur de la sombra
  \end{itemize}
  \begin{center}
  \includegraphics[scale=0.8]{./Figuras/s6.pdf}
  \end{center}
\onslide<7>
  \begin{itemize}
    \item Curva de la centralidad
  \end{itemize}
  \begin{center}
  \includegraphics[scale=0.8]{./Figuras/s7.pdf}
  \end{center}
\onslide<8>
  \begin{itemize}
    \item Curvas de simultaneidad de principio y fin del eclipse
  \end{itemize}
  \begin{center}
  \includegraphics[scale=0.8]{./Figuras/s8.pdf}
  \end{center}
\onslide<9>
  \vspace{0.8cm}
  \begin{center}
  \includegraphics[scale=0.4]{./Figuras/01ago2008.pdf}
  \end{center}
\onslide<10>
  \vspace{0.8cm}
  \begin{center}
  \includegraphics[scale=0.4]{./Figuras/31may2003.pdf}
  \end{center}
\onslide<11>
  \vspace{0.8cm}
  \begin{center}
  \includegraphics[scale=0.65]{./Figuras/19abr2004.pdf}
  \end{center}

\transblindshorizontal<9,10,11>
\end{overprint}
\end{frame}

%%%%%%%%%%%%%%%%%%%%%%%%%%% 9

\begin{frame}

\frametitle{Sistema fundamental de coordenadas}

\only<1,2,3>{
\begin{center}
\includegraphics[width=0.6\textwidth]{./Figuras/Sistemas.pdf}
\end{center}}

\begin{overprint}

\onslide<1>
\begin{itemize}
\item Definici\'{o}n
\item Plano y elipse fundamental
\item Plano del observador
\end{itemize}

\onslide<2>
Elementos besselianos
\begin{itemize}
\item Definici\'{o}n
\item $x, y, d, \mu, l_{p}, l_{s}, i_{p}, i_{s}$
\end{itemize}

\onslide<3>
Para pasar del sistema fundamental de coordenadas al geoc\'{e}ntrico, es preciso realizar un giro de $(d-90)$ grados en torno al eje $x$, llevando el eje $z$ sobre el $Z$, y de $(\mu-90)$ grados en torno al eje $Z$.
%\end{overprint}
%
%\begin{overprint}
\onslide<4>
Las coordenadas geogr\'{a}ficas cartesianas son:
\[
\begin{array}{r@{{}={}}l}
X & \rho\cos\phi\cos \lambda,\\
Y & \rho\cos\phi\sen \lambda,\\
Z & \rho\sen\phi.
\end{array}
\]
siendo $\rho$ la distancia geoc\'{e}ntrica, $\phi$ la latitud geoc\'{e}ntrica y $\lambda$ la longitud; que est\'{a}n ligadas a las geod\'{e}sicas mediante las siguientes f\'{o}rmulas:
\[
\rho  \cos\phi = (C + H) \cos\varphi, \qquad \rho  \sen\phi = (S + H) \sen\varphi,
\]
\[
C = \dfrac{1}{\sqrt{\cos^2\phi + (1 - f)^2 \sen^2\phi}}, \qquad S = C (1 - f)^2,
\]
\onslide<5>
Los cambios de coordenadas cartesianas de un sistema a otro quedan definidos por las siguientes relaciones matriciales:
\[
\left (
\begin{array}{c}
  X \\
  Y \\
  Z
\end{array}
\right )=
{\cal R}
\left (
\begin{array}{c}
  \xi \\
  \eta \\
  \zeta
\end{array}
\right ),\qquad
\left (
\begin{array}{c}
  \xi \\
  \eta \\
  \zeta
\end{array}
\right )=
{\cal R}^{\rm{T}}
\left (
\begin{array}{c}
  X \\
  Y \\
  Z
\end{array}
\right ),
\]
donde $\cal R$ es la matriz ortogonal
\[
{\cal R}=
\left (
\begin{array}{ccc}
  \sen\mu  & -\sen d\,\cos\mu & \phantom{-}\cos d\,\cos\mu \\
  \cos\mu & \phantom{-}\sen d\,\sen\mu & -\cos d\, \sen\mu \\
  0                & \cos d                 & \sen d                  \\
\end{array}
\right )
\]
y ${\cal R}^{\rm{T}}$ su transpuesta.
\onslide<6>
Las coordenadas en el sistema fundamental $(\xi,\eta,\zeta)$, de un punto de coordenadas geogr\'{a}ficas $(\lambda,\phi)$, ser\'an:
\[
\begin{array}{r@{{}={}}l}
\xi   & \rho \cos\phi \sen h, \\
\eta  & \rho \sen\phi \cos d - \rho \cos\phi\sen d \cos h, \\
\zeta & \rho \sen\phi \sen d + \rho \cos\phi\cos d \cos h,
\end{array}
\]
donde $h=\mu + \lambda$  es el horario local del eje de la sombra.

\vspace{0.5cm}

Sus derivadas con respecto al tiempo son:
\[
\begin{array}{r@{{}={}}l}
\xi' & \mu'(\zeta\cos d-\eta\sen d),\\
\eta' & -\zeta d'+\mu'\xi\sen d,\\
\zeta' & \eta d'-\mu'\xi\cos d.
\end{array}
\]

\onslide<7>
Para un observador de coordenadas $(\xi,\eta,\zeta)$, el radio del cono de penumbra en el plano del observador ($L_p$), del de sombra ($L_s$), as\'{\i} como la distancia ($\Delta$) al eje de la sombra, est\'an definidos por:
\begin{equation}\label{LpLsD}
\begin{array}{r@{{}={}}l}
L_{p} & l_{p}-\zeta i_{p},\\[1ex]
L_{s} & l_{s}-\zeta i_{s},\\[1ex]
\Delta & \sqrt{(x-\xi)^2+(y-\eta)^2}.
\end{array}
\end{equation}
\end{overprint}

\end{frame}

%%%%%%%%%%%%%%%%%%%%%%%%%%% 9
\section{Desarrollo del programa}
\subsection{Implementaci\'{o}n algoritmos}
%\begin{frame}
%\frametitle{C\'{a}lculo elementos besselianos}
%Sean $\vec{r}_{\astrosun}$ y $\vec{r}_{\leftmoon}$ los vectores de posici\'{o}n geoc\'{e}ntricos del Sol y la Luna,
%la direcci\'on del eje $z$ est\'a definida por el vector de posici\'on del Sol respecto a la Luna $\vec{r}_{\leftmoon\kern-.5ex\hbox{-\astrosun}} = \vec{r}_{\astrosun}-\vec{r}_{\leftmoon}$, quedando determinados los elementos besselianos $d$ y $\mu$.\newline
%Las coordenadas del centro de la Luna en el sistema fundamental definen los elementos $x$ e $y$:
%\[
%\left (
%\begin{array}{c}
%  x = x_{\leftmoon} \\
%  y = y_{\leftmoon} \\
%  z_{\leftmoon}
%\end{array}
%\right )=
%{\cal R}^{\rm{T}}
%\left (
%\begin{array}{c}
%  X_{\leftmoon} \\
%  Y_{\leftmoon} \\
%  Z_{\leftmoon}
%\end{array}
%\right ),
%\]
%Los elementos $l_{p}, l_{s}, i_{p}, i_{s}$ se obtienen por trigonometr\'{\i}a.
%\end{frame}

\begin{frame}
\frametitle{Otras par\'{a}metros necesarios}
Adem\'{a}s de los elementos besselianos se requieren dos funciones m\'{a}s:
\begin{itemize}
\item Valor del semieje de la elipse fundamental ($b$): Queda un\'{\i}vocamente determinado por el elemento $d$.
\item Distancia del eje de sombra a la elipse fundamental ($q$): Se obtiene una vez definida la elipse fundamental y sabiendo la posici\'{o}n de eje de sombra (elementos $x$ e $y$).
\end{itemize}
De todos los elementos besselianos, as\'{\i} como de estos nuevos par\'{a}metros se obtienen funciones interpolantes y sus derivadas mediante \texttt{Interpolation} y \texttt{Derivative}.
\end{frame}

\begin{frame}
\frametitle{Determinaci\'{o}n tipo de eclipse}
Se determinan en funci\'{o}n de si el m\'{\i}nimo de las funciones $q-l_{p}$, $q-l_{s}$, $q$, $q+l_{s}$ y $q+l_{p}$ es mayor o menor que 0.
\begin{center}
  \includegraphics[width=0.7\textwidth]{./Figuras/Sombras.pdf}
\end{center}
Se crean variables auxliares l\'{o}gicas.
\end{frame}

\begin{frame}
\frametitle{Par\'{a}metros para la conjunci\'{o}n en ascensi\'{o}n recta}
\end{frame}

\begin{frame}
\frametitle{Principio y fin de las distintas fases}
\end{frame}

\begin{frame}
\frametitle{M\'{a}ximo del eclipse}
\end{frame}

\begin{frame}
\frametitle{Magnitud del eclipse}
\end{frame}

\begin{frame}
\frametitle{Curva central del eclipse}
\end{frame}

\begin{frame}
\frametitle{Eclipse central al mediod\'{\i}a local}
\end{frame}

\begin{frame}
\frametitle{Curvas de contacto en el horizonte}
\end{frame}

\begin{frame}
\frametitle{Curvas de m\'{a}ximo en el horizonte}
\end{frame}

\begin{frame}
\frametitle{Curvas l\'{\i}mite Norte y Sur}
\end{frame} 

\begin{frame}
\frametitle{Puntos de coincidencia de las curvas de m\'{a}ximo en el horizonte con las l\'{\i}mite Norte y Sur}
\end{frame}

\begin{frame}
\frametitle{Curvas de simultaneidad de principio y fin de eclipse}
\end{frame}

\subsection{Dibujo del mapa}

\begin{frame}
\frametitle{Ordenado de puntos y tipo de representaci\'{o}n}
\end{frame}

 
\begin{frame}
\frametitle{Mapas}
\begin{overprint}
\only<1-2>{\center{Comparaci\'{o}n Mapas Almanaque y Generado}}

\only<1>{Eclipse anular de 26 de enero de 2009
\bigskip}

\only<2>{Eclipse parcial de 19 de abril de 2004
\bigskip}


\onslide<1>
\begin{center}
\includegraphics[width=0.45\textwidth]{./Figuras/EjemploMapaAnular.pdf}
\quad
\includegraphics[width=0.45\textwidth]{./Figuras/s1.pdf}\\
{\small Mapa almanaque \qquad \qquad \qquad \qquad Mapa programa \qquad}
\end{center}


\onslide<2>
\begin{center}
\includegraphics[height=0.4\textwidth,angle=-90]{./Figuras/EjemploMapaParcial2.pdf}
\qquad
\includegraphics[width=0.4\textwidth]{./Figuras/19abr2004.pdf}\\
\end{center}

\end{overprint}

\end{frame}
%%%%%%%%%%%%%%%%%%%%%%%%%%% 10



\begin{frame}

\frametitle{Algunas observaciones finales}

\begin{itemize}
\item El formato se puede cambiar. V\'{e}anse las primeras l\'{\i}neas del documento \LaTeX.
\pause
\item El t\'{\i}tulo, la p\'{a}gina del \'{\i}ndice y la cabecera y el pie de p\'{a}gina de cada pantalla se forman de manera autom\'{a}tica.
\pause
\item Adem\'{a}s de las \'{o}rdenes normales de \LaTeX, las que se han usado en esta presentaci\'{o}n son: \texttt{frame, frametitle, alert, <n-m>, only, block, alertblock, fbox, overprint, onslide, pause, movie}.
\pause
\item Se puede emplear \texttt{latex} o \texttt{pdflatex}. A veces es necesario componer el documento varias veces hasta que hacen efecto los cambios.
\end{itemize}

\end{frame}

\begin{frame}

%%%%%%%%%%%%%%%%%%%%%%%%%%% 11

\frametitle{M\'{a}s observaciones finales}

\begin{itemize}
\item El paquete �beamer� ha sido desarrollado por Till Tantau (el mismo que hace \texttt{tikz-pgf}), funciona en cualquier ordenador y es totalmente gratis (como \LaTeX).
\pause
\item Se puede coger en la p\'{a}gina \url{http://latex-beamer.sourceforge.net/}.
\pause
\item Las distribuciones de \TeX\ actuales ya lo incluyen.
\pause
\item Tiene un voluminoso manual con muchos ejemplos.
\pause
\item Viene con modelos de presentaciones que sirven como
�esqueleto� de partida para la nuestra.
\pause
\item Tambi\'{e}n podemos partir de alguna creada previamente por
un amigo. \pause Como \'{e}sta, que proviene de una de \alert{Mario P\'{e}rez}.
\end{itemize}

\begin{center}
Espero que os haya gustado, \textbf{!`y que lo us\'{e}is!}\pause

\medskip
\alert{FIN}
\end{center}

\end{frame}

\end{document}

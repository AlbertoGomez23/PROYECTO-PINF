\documentclass[10pt, a4paper]{article}
\usepackage[spanish]{babel}            %paquete para acentos forma estandar en espa\~{n}ol.
\usepackage[ansinew]{inputenc}         %paquete solo para Windows para reconocer acentos.
\usepackage{graphicx,color}            %paquete para manejar figuras y color de las mismas (incluye letras).
\usepackage{lscape}
%\usepackage{subfigure}
\usepackage{amsmath,amsfonts,amssymb}  %paquete de simbolos matem\'{a}ticos American Mathematical Socety
\usepackage{latexsym}                  % Paquete que ofrece adicional simbolos matem\'{a}ticos

%\usepackage[latin1]{inputenc}          %paquete solo para Unix para reconocer acentos.

%**********************************************************************************************************
% Trabajo de Programaci\'{o}n No Lineal. Master Modelizaci\'{o}n Matem\'{a}tica, Estad\'{\i}stica
% y Computaci\'{o}n. Periodo 2009-2010.
%**********************************************************************************************************

\title{\textbf{TRABAJO F\'{I}N DE M\'{A}STER: OBTENCI\'{O}N Y DIBUJO DE CURVAS REPRESENTATIVAS DE UN ECLIPSE SOLAR}}
\author{Federico Baeza Richer}
\date{XX de junio de 2010}
%\emph{baezafede@gmail.com} colocar correo electr\'{o}nico.

\begin{document}

\maketitle
\newpage

\begin{abstract}
Se presenta el trabajo f\'{\i}n de m\'{a}ster de t\'{\i}tulo Obtenci\'{o}n y dibujo de las curvas representativas de un eclipse solar, correspondiente al M\'{a}ster en Modelizaci\'{o}n Matem\'{a}tica, Estad\'{\i}stica y Computaci\'{o}n (curso 2009-2010). El trabajo, como su nombre indica, consiste en la obtenci\'{o}n de todos los par\'{a}metros y elementos caracter\'{\i}sticos de un eclipse solar, y del posterior dibujo sobre la esfera terrestre, de todas las curvas representativas.  El trabajo fue dirigido por Teodoro L\'{o}pez Moratalla (Jefe de la Secci\'{o}n de Efem\'{e}rides del Real Instituto y Observatorio de la Armada), como responsable de la empresa; y por Alberto Abad (Profesor de Astrodin\'{a}mica de la Universidad de Zaragoza) como responsable del m\'{a}ster. \\
\end{abstract}
\newpage

\section{Generalidades sobre Eclipses}
El t\'{e}rmino eclipse se aplica indistintamente a dos fen\'{o}menos
provocados por las posiciones relativas de tres astros, el Sol
(emisor luminoso) y la Tierra junto con la Luna (cuerpos opacos que
interceptan la luz solar).\\

Un eclipse de Sol se produce cuando este astro es ocultado
por el globo de la Luna, interponi\'{e}ndose entre la Tierra y el
Sol. Un eclipse de Sol tiene lugar siempre en fase de Luna Nueva
siendo \'{e}sta una condici\'{o}n necesaria pero no suficiente para
que se produzca el fen\'{o}meno.\\

La condici\'{o}n de Luna Nueva ser\'{\i}a suficiente si las
\'{o}rbitas del Sol y de la Luna fueran coplanarias, hecho que no
ocurre al ser su inclinaci\'{o}n relativa pr\'{o}xima a los
$5^\circ$, por lo tanto la Luna se encuentra en la mayor\'{\i}a de
las ocasiones por debajo o encima de la ecl\'{\i}ptica.\\

Para que se produzca un eclipse de Sol, la Luna debe hallarse en
el plano de la ecl\'{\i}ptica o en uno muy cercano, en fase de Luna
Nueva. Es decir la Luna debe encontrarse en las proximidades de un
nodo.\\

Los eclipses de Luna est\'{a}n determinados por
el paso de nuestro sat\'{e}lite por la sombra de la Tierra y siempre
suceden en fase de Luna Llena y al igual que en los eclipses
solares cuando la Luna se encuentra en el nodo o en sus
proximidades.\\


\newpage
\section{Tipos de Eclipses de Sol}
\textbf{Definici\'{o}n:} Se llama cono de sombra al cono formado por las tangentes exteriores del Sol y la Luna y cono de penumbra al formado por las tangentes interiores. El eje com\'{u}n a ambos conos, materializado por la l\'{\i}nea que une los centros \'{o}pticos de Sol y Luna, se denomina eje de la sombra.\\

\begin{figure}[ht!]
\centering
\includegraphics[scale=0.50]{./Figuras/Sombra.pdf}\\
\caption{Cono de Sombra}\label{f1}
\end{figure}

\begin{figure}[ht!]
\centering
\includegraphics[scale=0.50]{./Figuras/Penumbra.pdf}\\
\caption{Cono de Penumbra}\label{f2}
\end{figure}

Se produce un eclipse de Sol cuando el cono de penumbra alcanza alg\'{u}n punto de la superficie terrestre. Seg\'{u}n la parte de la Tierra que quede inmersa dentro de los respectivos conos, los eclipses se dividen tradicionalmente en parciales, no centrales y centrales.\\

Eclipses Parciales son aquellos en los que el cono de sombra no alcanza ning\'{u}n punto de la superficies terrestre. Estos eclipses siempre se producen en latitudes altas (norte o sur).\\

\begin{figure}[ht!]
\centering
\includegraphics[scale=0.50]{./Figuras/A1.pdf}\\
\caption{Eclipse Parcial}\label{A1}
\end{figure}


Eclipses no Centrales son aquellos en los que el cono de sombra s\'{\i} alcanza alg\'{u}n punto de la superficies terrestre, pero no as\'{\i} el eje de la sombra. Este tipo de eclipses afecta siempre a regiones polares.\\

\begin{figure}[ht!]
\centering
\includegraphics[scale=0.50]{./Figuras/B1.pdf}\\
\caption{Eclipse no Central}\label{B1}
\end{figure}

Eclipses Centrales son aquellos en los que el cono de sombra s\'{\i} alcanza alg\'{u}n punto de la superficies terrestre.\\

Para este trabajo, y por las implicaciones geom\'{e}tricas y de programaci\'{o}n que veremos m\'{a}s adelante, distinguiremos adem\'{a}s tres tipos distintos dentro de los Eclipses Centrales. \\

Eclipses centrales con una sola curva de totalidad, cuando el cono de sombra nunca est\'{a} totalmente inmerso en la superficie terrestre.\\

\begin{figure}[ht!]
\centering
\includegraphics[scale=0.50]{./Figuras/C1.pdf}\\
\caption{Eclipse Central con una sola curva de totalidad}\label{C1}
\end{figure}

Eclipses centrales con una sola curva de parcialidad, cuando el cono de sombra s\'{\i} llega a estar totalmente inmerso en la superficie terrestre, pero no as\'{\i} el de penumbra.\\

\begin{figure}[ht!]
\centering
\includegraphics[scale=0.50]{./Figuras/D1.pdf}\\
\caption{Eclipse Central con una sola curva de parcialidad}\label{D1}
\end{figure}


Eclipses centrales con ambas curvas de parcialidad, cuando el cono de penumbra, en alg\'{u}n momento s\'{\i} llega a estar totalmente inmerso en la superficie terrestre. \\

\begin{figure}[ht!]
\centering
\includegraphics[scale=0.50]{./Figuras/E1.pdf}\\
\caption{Eclipse Central con ambas curvas de parcialidad}\label{E1}
\end{figure}

Dentro de los eclipses centrales, a su vez se subdividen en anulares, totales y totales-anulares.\\

Cuando el di\'{a}metro lunar es menor que el solar, esto ocurre cuando
la Luna se encuentra en las proximidades del apogeo, \'{e}sta no cubre
totalmente el disco solar y el eclipse se llama anular.\\

\begin{figure}[ht!]
\centering
\includegraphics[scale=0.50]{./Figuras/Anular.pdf}\\
\caption{Eclipse Anular}\label{Anular}
\end{figure}

Cuando, por el contrario, el di\'{a}metro lunar es mayor que el solar, el eclipse se llama total. Esto ocurre cuando la Luna se encuentra en las proximidades del perigeo y no cubre totalmente el disco solar.\\

\begin{figure}[ht!]
\centering
\includegraphics[scale=0.50]{./Figuras/Total.pdf}\\
\caption{Eclipse Total}\label{Total}
\end{figure}

Cuando en el alg\'{u}n punto de la Tierra el eclipse se ve como anular y en alg\'{u}n otro como total, se le llama total-anular.


\newpage
\section{Par\'{a}metros y curvas caracter\'{\i}sticas de los eclipses solares}

\subsection{Datos y par\'{a}metros descriptivos del eclipse}
\begin{itemize}
  \item Tipo de eclipse seg\'{u}n la parte de la Tierra que queda en los distintos conos: Parcial, no Central, Central una sola curva de totalidad, Central con una sola curva de parcialidad y Central con ambas curvas de parcialidad.
  \item Para los eclipses centrales distinguir adem\'{a}s si son totales, anulares o totales-anulares.
  \item Magnitud del eclipse. Se mide en el m\'{a}ximo del eclipse. Para los eclipses parciales y totales es la distancia entre el borde del Sol m\'{a}s pr\'{o}ximo al centro de la Luna y el borde de la Luna m\'{a}s pr\'{o}ximo al centro del Sol, dividido por el di\'{a}metro solar. En los anulares es la relaci\'{o}n entre los di\'{a}metros de la Luna y del Sol.
\end{itemize}



\subsection{Par\'{a}metros en conjunci\'{o}n en Ascensi\'{o}n Recta}
Son los siguientes datos, para el instante en que el Sol y la Luna tienen la misma ascensi\'{o}n recta.
\begin{itemize}
  \item Hora de la conjunci\'{o}n.
  \item Declinaci\'{o}n de Sol y Luna.
  \item Paralaje horizontal ecuatorial de Sol y Luna (hacen referencia a la distancia a la Tierra).
  \item Semidi\'{a}metro verdadero de Sol y Luna (hacen referencia al tama\~{n}o visual).
\end{itemize}

\subsection{}
\subsection{Instantes y puntos caracter\'{\i}sticos}
\begin{itemize}
  \item Principio y fin del eclipse parcial.
  \item Principio y fin del eclipse total/anular.
  \item Principio y fin de la curva central del eclipse.
  \item Contactos interiores del cono de sombra.
  \item Contactos interiores del cono de penumbra.
  \item M\'{a}ximo del eclipse. Es el instante de m\'{a}xima proximidad del eje de la sombra al centro de la Tierra. Para un eclipse central el punto de m\'{a}ximo es el de corte del eje con la Tierra en ese instante. Para un eclipse no central es el punto de la Tierra m\'{a}s cercano al eje en ese instante (l\'{o}gicamente es un punto con su m\'{a}ximo de eclipse en el horizonte).
  \item Eclipse central al mediod\'{\i}a. Punto de la curva central que tiene el m\'{a}ximo del eclipse al Norte o al Sur.
  \item Coincidencia de las curvas de m\'{a}ximo en el horizonte con las curvas l\'{\i}mite Norte/Sur sombra/penumbra.
\end{itemize}

\subsection{Curva de eclipse central}
Es el lugar geom\'{e}trico de los puntos de corte del eje de la sombra con la Tierra.

\subsection{Curvas de contacto en el horizonte}
Son las cuatro curvas siguientes, formadas por los puntos a los que les ocurre lo que se explica a continuaci\'{o}n:
\begin{itemize}
  \item Fin del eclipse al orto: En el momento en que aparece el Sol por el horizonte finaliza el eclipse. Es el l\'{\i}mite occidental del eclipse.
  \item Inicio del eclipse al orto: En el momento en que aparece el Sol por el horizonte comienza el eclipse. Es el l\'{\i}mite occidental de los puntos que pueden ver todo el eclipse por encima del horizonte.
  \item Fin del eclipse al ocaso: Coincide el ocaso con el fin del eclipse. Es el l\'{\i}mite oriental de los puntos que pueden ver todo el eclipse por encima del horizonte.
  \item Inicio del eclipse ocaso: En el momento del ocaso ocurre el inicio del eclipse. Es el l\'{\i}mite oriental de visualizaci\'{o}n del eclipse.
\end{itemize}

\subsection{Curvas de m\'{a}ximo de eclipse en el horizonte}
El instante de m\'{a}ximo del eclipse, m\'{a}xima ocultaci\'{o}n del disco solar por parte de la Luna, se produce cuando el Sol est\'{a} en el horizonte. L\'{o}gicamente hay dos curvas:
\begin{itemize}
  \item Curva de m\'{a}ximo al orto: Al orto es el m\'{a}ximo del eclipse y conforme sale el Sol va disminuyendo el eclipse. Esta curva se encuentra entre las de inicio y fin del eclipse al orto.
  \item Curva de m\'{a}ximo al ocaso: Comienza el eclipse y va aumentando conforme va bajando el Sol coincidiendo el ocaso con el m\'{a}ximo. De forma an\'{a}loga esta curva se encuentra entre las de inicio y fin del eclipse al ocaso.
\end{itemize}

\subsection{Curvas l\'{\i}mite Norte y Sur}
Puntos de la Tierra m\'{a}s al norte y m\'{a}s al sur desde donde se puede ver el eclipse.

\subsection{Curvas instant\'{a}neas del eclipse}
Cortes del cono de penumbra con la Tierra para un instante determinado. Incluye l\'{o}gicamente los puntos para los que empieza el eclipse en ese momento, para los que termina, as\'{\i} como lo l\'{\i}mites Norte y Sur para los que el eclipse consiste en un solo punto instant\'{a}neo.

\newpage

\section{Sistema Fundamental de Coordenadas}

Es un sistema geoc\'{e}ntrico. El eje $z$ es paralelo al eje de sombra, positivo en el sentido Tierra-Luna. El plano fundamental es el perpendicular al eje z por el origen y en el se sit\'{u}an los ejes $x$ e $y$. El eje $x$ es la intersecci\'{o}n del ecuador con el plano fundamental, positivo hacia el Este. El eje $y$ forma un triedro directo con los otros dos. El plano del observador es el plano paralelo al fundamental que pasa por la situaci\'{o}n del observador. Las coordenadas de los puntos en este sistema de referencia se denotar\'{a}n por $\xi$, $\eta$ y $\zeta$.

\begin{figure}[ht!]
\centering
\includegraphics[scale=0.60]{./Figuras/Sistemas.pdf}\\
\caption{Sistema fundamental de coordenadas}\label{eleB}
\end{figure}

\newpage
\section{Elementos Besselianos}
Los elementos besselianos son un conjunto de funciones dependientes del tiempo que describen la geometr\'{\i}a del eclipse. Estos elementos son:
\begin{itemize}
  \item \emph{x, y}: Coordenadas, en el sistema fundamental, de la intersecci\'{o}n del eje de la sombra con el plano fundamental. Se dan en unidades de radio ecuatorial terrestre.
  \item \emph{d, $\mu$}: Declinaci\'{o}n y \'{a}ngulo horario en Greenwich del eje de la sombra (y del eje z). Se dan en grados sexagesimales.
  \item \emph{$l_{p}, l_{s}$}: Radios, en unidades del radio ecuatorial terrestre, de las respectivas intersecciones de los conos de sombra y penumbra con el plano fundamental. Por convenio, estos radios tienen el mismo signo que la coordenada $z$ del v\'{e}rtice del cono correspondiente. As\'{\i}, $l_{p}$ es siempre positivo, mientras que $l_{s}$ puede ser positivo, negativo o nulo.
  \item \emph{$i_{p}, i_{s}$}:
\end{itemize}

\newpage

\section{M\'{e}todos cl\'{a}sicos frente a m\'{e}todo utilizado}

\newpage

\section{Datos utilizados para los eclipses y su lectura}
Datos del JPL

\newpage
\section{Funciones auxiliares utilizadas}
\subsection{Formatos de horas, grados y radianes}
\begin{itemize}
  \item \emph{hms2h}: Paso de horas, minutos y segundos a horas.
  \item \emph{h2hms}: Paso de horas a horas, minutos y segundos.
  \item \emph{gms2rad}: Paso de grados, minutos y segundos a radianes.
  \item \emph{rad2gms}: Paso de radianes a grados, minutos y segundos.
  \item \emph{grad2gm}: Paso de grados a grados y minutos.
  \item \emph{hms2rad}: Paso de horas, minutos y segundos a radianes.
  \item \emph{rad2hms}: Paso de radianes a horas, minutos y segundos.
  \item \emph{hms2gms}: Paso de horas, minutos y segundos a grados, minutos y segundos.
  \item \emph{gms2hms}: Paso de grados, minutos y segundos a horas, minutos y segundos.
  \item \emph{h2hms}: Paso de horas a minutos y segundos.
\end{itemize}
\subsection{Fechas y tiempos}
\begin{itemize}
  \item \emph{diajuliano}: A partir del d\'{\i}a, mes y a\~{n}o se obtiene el d\'{\i}a juliano.
  \item \emph{tsidmedloc}: Da el tiempo sid\'{e}reo medio local, a partir del d\'{\i}a, mes, a\~{n}o, hora UT y longitud.
  \item \emph{cormerefe, coramerefe}: Correcci\'{o}n en grados de longitud debido a la diferencia entre el Tiempo Terrestre (en el que figuran los datos) y el Tiempo Universal Coordinado (por el que nos regimos).
\end{itemize}
\subsection{Transformaci\'{o}n de coordenadas}
\begin{itemize}
  \item \emph{ecu2cart}: Paso de coordenadas polares a cartesianas en sistema ecuatorial.
  \item \emph{cart2ecu}: Paso de coordenadas cartesianas a polares en sistema ecuatorial.
  \item \emph{Rx}: Rotaci\'{o}n en torno a \emph{eje x}.
  \item \emph{Ry}: Rotaci\'{o}n en torno a \emph{eje y}.
  \item \emph{Rz}: Rotaci\'{o}n en torno a \emph{eje z}.
  \item \emph{Matcambcoor}: Matriz de cambio de coordenadas de sistema fundamental a ecuatorial (referido a Greenwich) dados los elementos besselianos $d$ y $\mu$.
  \item \emph{matcambcoor}: Matriz de cambio de coordenadas de sistema fundamental a ecuatorial (referido a Greenwich) para un instante determinado.
  \item \emph{cambcoorcart}: Paso de coordenadas cartesianas de sistema fundamental a ecuatorial (referido a Greenwich) para un instante determinado.
  \item \emph{geodes2geocen}: Paso latitud geod\'{e}sicas y altura a latitud geoc\'{e}ntrica y distancia radial.
  \item \emph{latgeocen2latgeode}: Paso de latitud geoc\'{e}ntrica a geod\'{e}sica.
  \item \emph{ecu2ecli}: Paso de coordenadas polares de sistema ecuatorial a ecl\'{\i}ptico.
  \item \emph{ecli2ecu}: Paso de coordenadas polares de sistema ecl\'{\i}ptico a ecuatorial.
  \item \emph{correcluna}: Correcci\'{o}n de coordenadas de la Luna por diferencia entre centro de masas y centro geom\'{e}trico.
\end{itemize}
\subsection{Funciones geom\'{e}tricas del elipsoide}
\begin{itemize}
  \item \emph{radeliphor}: Valor del semieje menor de la elipse fundamental (corte de plano fundamental con elipsoide terrestre) para cada valor del \emph{elemento d}.
  \item \emph{disEjeElip}: Distancia con signo (positiva exterior, negativa interior) de un punto a una elipse.
\end{itemize}
\subsection{Coordenadas geogr\'{a}ficas en cada instante de punto de elipse fundamental m\'{a}s pr\'{o}ximo a eje de sombra}
\begin{itemize}
  \item $\xi \eta fun$: Coordenadas fundamentales de punto elipse fundamental m\'{a}s pr\'{o}ximo a eje de sombra.
  \item $\xi \eta fun2xyz$: Coordenadas cartesianas en sistema ecuatorial referido a Greenwich.
  \item $\xi \eta fun2ecu$: Coordenadas polares en sistema ecuatorial referido a Greenwich.
  \item $\xi \eta fun2lonlat$: Coordenadas geogr\'{a}ficas de dicho punto.
\end{itemize}
\subsection{Coordenadas en sistema fundamental en cada instante, de cualquier punto del elipsoide}
\begin{itemize}
  \item \emph{lonlatt2xyz, lonlatth2xyz}: Coordenadas cartesianas ecuatoriales (referidas a Greenwich) de cualquier punto terrestre (sin o con elevaci\'{o}n respecto a la superficie).
  \item $lonlatt2\xi \eta \zeta, lonlatth2\xi \eta \zeta$: Idem pero ya en coordenadas fundamentales.
\end{itemize}
\subsection{Funciones corte eje de sombra con elipsoide y duraci\'{o}n totalidad en eje}
\begin{itemize}
  \item $corteejeelip2\xi \eta \zeta$: Coordenadas en sistema fundamental del punto de corte de eje de sombra con elipsoide, dados los elementos besselianos.
  \item \emph{corteejeelip2xyz}: Coordenadas en sistema eucatorial referido a Greenwich del punto de corte.
  \item \emph{corteejeelip2lonlat}: Coordenadas geogr\'{a}ficas del punto de corte.
  \item \emph{corteejeelip}: Coordenadas geogr\'{a}ficas del punto de corte para un instante determinado.
  \item \emph{deltamenosL}: Da la distancia de un punto al borde del cono de sombra para un instante determinado.
  \item \emph{deltamenosLt}: Igual que la anterior pero los valores para un intervalo de tiempos en torno al instante actual.
  \item \emph{duraciontoteje}: Da el tiempo (en minutos y segundos) de totalidad de un punto en torno a un instante determinado.
  \item \emph{puntoejedur}: Igual que el anterior pero devolviendo adem\'{a}s la hora y la posici\'{o}n.
\end{itemize}
\subsection{Funciones corte cono de penumbra con elipse fundamental}
\begin{itemize}
 \item \emph{cortehor}: Coordenadas en sistema fundamental de los puntos de corte del cono de penumbra con elipse fundamental, dados los elementos besselianos.
  \item \emph{cortehor2xyz}: Coordenadas en sistema eucatorial referido a Greenwich de los puntos de corte.
  \item \emph{cortehor2lonlat}: Coordenadas geogr\'{a}ficas de los puntos de corte.
  \item \emph{cortehor2lonlatt}: Coordenadas geogr\'{a}ficas de los puntos de corte para un instante determinado.
\end{itemize}
\subsection{Funciones corte cono de penumbra con elipsoide}
\begin{itemize}
  \item $corteconopelip2\xi \eta \zeta$: Coordenadas en sistema fundamental de los puntos de corte del cono de penumbra con elipsoide, dados los elementos besselianos.
  \item \emph{corteconopelip2xyz}: Coordenadas en sistema eucatorial referido a Greenwich de los puntos de corte.
  \item \emph{corteconopelip2lonlat}: Coordenadas geogr\'{a}ficas de los puntos de corte.
  \item \emph{corteconopelip}: Coordenadas geogr\'{a}ficas de los puntos de corte para un instante determinado.
\end{itemize}
\subsection{Funciones curvas l\'{\i}mite Norte y Sur por tangentear cono}
\begin{itemize}
  \item $elxelylpipd\mu 2\zeta Q$: Coordenadas en sistema fundamental de los puntos l\'{\i}mite, dados los elementos besselianos.
  \item $elxelylpipd\mu2xyz$: Coordenadas en sistema eucatorial referido a Greenwich de los puntos l\'{\i}mite.
  \item $elxelylpipd\mu2lonlat$: Coordenadas geogr\'{a}ficas de los puntos de l\'{\i}mite.
  \item \emph{puntoslimitep}: Coordenadas geogr\'{a}ficas de los puntos l\'{\i}mite del cono de penumbra para un instante determinado.
  \item \emph{puntoslimites}: Coordenadas geogr\'{a}ficas de los puntos l\'{\i}mite del cono de sombra para un instante determinado.
\end{itemize}
\subsection{Funciones curvas l\'{\i}mite Norte y Sur por estar el eclipse bajo el horizonte}
\begin{itemize}
  \item $climite2\xi \eta \zeta$: Coordenadas en sistema fundamental de los puntos l\'{\i}mite, dados los elementos besselianos.
  \item \emph{climite2xyz}: Coordenadas en sistema eucatorial referido a Greenwich de los puntos l\'{\i}mite.
  \item \emph{climite2lonlat}: Coordenadas geogr\'{a}ficas de los puntos de l\'{\i}mite.
  \item \emph{climite}: Coordenadas geogr\'{a}ficas de los puntos l\'{\i}mite para un instante determinado.
\end{itemize}
\subsection{Funciones curvas de m\'{a}ximo de eclipse al orto/ocaso}
\begin{itemize}
  \item $maxhor2\xi \eta \zeta$: Coordenadas en sistema fundamental de los puntos de m\'{a}ximo en el horizonte, dados los elementos besselianos.
  \item \emph{maxhor2xyz}: Coordenadas en sistema eucatorial referido a Greenwich de los puntos de m\'{a}ximo en el horizonte.
  \item \emph{maxhor2lonlat}: Coordenadas geogr\'{a}ficas de los puntos de l\'{\i}mite.
  \item \emph{maxhor}: Coordenadas geogr\'{a}ficas de los puntos de m\'{a}ximo en el horizonte para un instante determinado.
\end{itemize}
\subsection{Funciones para hallar los puntos de corte de las curvas de m\'{a}ximo en el horizonte con las l\'{\i}mites Norte y Sur de conos de penumbra y de sombra}
\begin{itemize}
  \item
  \item
  \item
  \item
  \item
\end{itemize}

\newpage
\section{Simplificaciones utilizadas}
\begin{itemize}
  \item Forma de La Tierra elipsoide de revoluci\'{o}n con par\'{a}metros definidos por IAU (1976).
  \item Formas Sol y Luna esferas con radio definido por IAU (1976).
  \item Se considera que el eclipse est\'{a} en el horizonte, es decir que se est\'{a} produciendo al orto o al ocaso, cuando el punto de observaci\'{o}n tiene nula su componente $\zeta$ en coordenadas fundamentales. El error m\'{a}ximo geom\'{e}trico de considerarlo as\'{\i} es, en el pero de los casos aproximadamente $16'$ de arco (semiapertura del cono de penumbra). Tampoco se considera la refracci\'{o}n atmosf\'{e}rica, que en el horizonte puede alcanzar los $32'$ de arco y que es muy dif\'{\i}cil de modelizar. Si se quisiese ser m\'{a}s preciso habr\'{\i}a que estudiar las condiciones locales del eclipse.
  \item
  \item
  \item
  \item
  \item
  \item
  \item
  \item
  \item
  \item
  \item
  \item
\end{itemize}

\newpage
\section{Algoritmos obtenci\'{o}n de puntos y curvas}
\subsection{Distancias del eje de sombra a la elipse fundamental}
Se calcula para cada instante, la distancia de eje de sombra a la elipse fundamental, consider\'{a}ndola positiva si el eje est\'{a} fuera de la eclipse y negativa si se encuentra en el interior. Este valor, por s\'{\i} mismo, o comparado con el radio de los conos de sombra/penumbra, permiten saber si hay curva central y si hay contactos exteriores o interiores.
\subsection{Instantes y Puntos de contacto con elipsoide}
Se calcula el instante para el cual la distancia desde el eje de sombra a la elipse del plano fundamental, coincide con el radio del cono de sombra/penumbra en el plano fundamental. Se calcula para dichos instantes los puntos de tangencia y se obtiene posici\'{o}n geogr\'{a}fica.
\subsection{Curva central del Eclipse}
Se calcula para distintos instantes el punto de corte del eje de sombra con el elipsoide. Se obtienen l\'{o}gicamente dos puntos y se desecha el de $seta<0$.
\subsection{Curva de contacto en el horizonte}
Son, para cada instante, los puntos de la elipse fundamental, para los cuales la distancia al eje de sombra coincide con el radio del cono en el plano fundamental. Se desechan las soluciones no reales.
\subsection{Curva de m\'{a}ximo en el horizonte}
Son los puntos de la elipse fundamental, para los cuales la funci\'{o}n $l-delta$ tiene un m\'{a}ximo. De las soluciones obtenidas se mantienen exclusivamente las de puntos reales y que adem\'{a}s $l-delta>0$.
\subsection{Curvas l\'{\i}mite Norte y Sur}
Se distinguen dos casos distintos, por un lado los puntos, que estando sobre el horizonte, s\'{o}lo alcanzan a ver la tangencia de Sol y Luna, y por otro los puntos que estando inmersos en el cono de sombra/penumbra, tienen el eclipse por debajo del horizonte y s\'{o}lo alcanzan a llegar a $seta=0$.
Los primeros son puntos a la vez del elipsoide y del cono ($delta = L$), para los cuales la funci\'{o}n $L-delta$ tiene un m\'{a}ximo. Por facilidad computacional y por ser equivalente en este caso, se maximiza la funci\'{o}n $L^2-delta^2$. Se eliminan las soluciones imaginarias y las que dan un valor de $seta<0$.
Para el segundo caso se calculan los puntos que maximizan la funci\'{o}n $seta$ y que adem\'{a}s cumplen $seta=0$. Se mantienen s\'{o}lo las soluciones reales y para las que $l-delta>0$.
\subsection{Curvas instant\'{a}neas del eclipse}
Son los puntos, para cada instante, de corte del cono de penumbra con el elipsoide terrestre. Lo que se hace es para distintos $seta>0$, los puntos de corte del cono con la elipse de cada plano. Se mantienen s\'{o}lo las soluciones reales.

\newpage
\section{Dibujo en Planisferio}


\newpage
\begin{thebibliography}{99}
\bibitem a W. M. SMART, (1977): Textbook on Spherical Astronomy.
\bibitem a F. JAVIER GIL CHICA (1996): Teor\'{\i}a de eclipses, ocultaciones y tr\'{a}nsitos.
\bibitem a JEAN MEEUS, (1989): Elements of Solar Eclipses 1951-2200.
\bibitem a BAO-LIN LIU, ALAN D. FIALA (1992): Canon of Lunar Eclipses 1500
B.C.- A.D. 3000.
\bibitem a ORUS, J.J., CATALA, M.A., (1987)Apuntes de Astronom\'{\i}a Tomo I.
\bibitem a F. ESPENAK, (1997): Total Solar Eclipse of 1999 August 11. ed.
NASA (RP1398).
\bibitem a DWIGHT ENNIS, (2000): Eclipses and the Moon's Nodes.
http://www.astrologyclub.org/articles/nodes/nodes.htm
\bibitem a JUAN CARLOS C., MIQUEL S.-R.(2003): Unidad Did\'{a}ctica Eclipses.
http://www.fecyt.es/semanadelaciencia2003/eclipse/pdf/UDE.pdf

\end{thebibliography}

\end{document}
